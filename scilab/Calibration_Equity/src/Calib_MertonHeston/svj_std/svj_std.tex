\documentclass[12pt,a4paper]{article}
\usepackage{color}
\def\dps{\displaystyle}
\def\N{\mathcal{N}}
%
\title{Calibration of Stochastic Volatility model with Jumps}
\author{A. Ben Haj Yedder}
\begin{document}
\maketitle
%
The evolution process of the Heston model, for the stochastic
volatility,  and Merton model, for the jumps, is:
%
\[
\left\{
\begin{array}{lll}
\dps \frac{d S_t}{S_t} &=& (r-d) dt + \sqrt{V_t} dW_t^1 + (e^J - 1) dN_t\\
\dps  d V_t &=& \kappa (\theta - V_t) dt + \sigma_v \sqrt{V_t} dW_t^2  \\
S(t=0) &=& S_0 \\
V(t=0) &=& V_0
\end{array}
\right.
\]
 where $d<W^1,W^2>_t = \rho dt$ and $J \sim \N(m,v)$.

For European options, two pricing formula are giving based on the
Fourier transform  method \cite{sepp}. In this document, we use
the following notaton: $\varphi=+1$ for a call and $\varphi=-1$
for a put; $\tau=T-t$; $x_t=\ln(S_t)$ and $X = ln(S_t/K) +
(r-d)\tau $.
%

\section{The characteristic formula}\label{SecCF}
The price $F(x_t,t)$ is given by:
%\footnote{In \cite{sepp}, formula (3.7) the term $e^{-(T-t)r}$, before
%  $f(x,V,\lambda,t)$ was forgotten.}:
\[F(x_t,t)=\frac{1+\varphi}{2}e^{x_t-(T-t)d}+\frac{1-\varphi}{2}e^{1-(T-t)r}K - e^{-(T-t)r}f(x,V,\lambda,t)\]
where
\[f(x,V,\lambda,\tau)=\frac{K}{\pi}\int_0^\infty\Re\left[ \frac{Q(k,x,V,\lambda,\tau)}{k^2+1/4}\right] dk\]
with
\[Q(k,x,V,\lambda,\tau) = e^{(-ik+1/2)X+ A(k,\tau) + B(k,\tau) V_0
+ C(k,\tau) + D(k,\tau)\lambda}
\]
The hedge $\delta$ is given by:
\[
\delta = \frac{1+\varphi}{2}e^{-(T-t)d} - e^{-(T-t)r}
\frac{K}{\pi}\int_0^\infty\Re\left[\frac{(1/2-ik)}{S_t}
\dot  \frac{Q(k,x,V,\lambda,\tau)}{k^2+1/4}\right] dk 
\]
The coefficients  $A(k,\tau)$, $B(k,\tau)$, $C(k,\tau)$ and
$D(k,\tau)$ are specified as follows:
\begin{enumerate}
\item Volatility: \\
\begin{itemize}
\item Constant volatility : \\
$A(k,\tau)=0$, $B(k,\tau)=-1/2(k^2+1/4)\tau$.
\item Stochastic
volatility (Heston) :\\
$\dps A(k,\tau)=-\frac{\kappa\theta}{\sigma_v^2}\left[\psi_+ \tau
+
    2ln\left( \frac{\psi_-+\psi_+e^{-\tau\zeta}}{2\zeta}\right) \right] $ \\
 $\dps B(k,\tau)=-(k^2+1/4)\frac{1-e^{-\tau\zeta}}{\psi_-+\psi_+e^{-\tau\zeta}} $ \\
where $\dps \psi_\pm = \mp (u+ik\rho\sigma_v) + \zeta$, $\dps
\zeta=\sqrt{k^2\sigma_v^2(1-\rho^2) + 2ik\rho\sigma_vu
+u^2+\sigma_v^2/4}$ and $u=\kappa-\rho\sigma_v/2$.
\end{itemize}
%
\item Jumps :\\
\begin{itemize}
\item Merton model : constant jump rate intensity and log-normal jump size distribution \\
$C(k,\tau)=0$, $D(k,\tau)=\tau\Lambda(k)$ where \\
$\dps \Lambda(k) = e^{-ik(m+v^2/2) - (k^2-1/4)v^2/2 + 1/2m)} -1 -
(-ik+1/2)(e^{m+v^2/2}-1)$.
\end{itemize}
\end{enumerate}
%
\section{The Black-Scholes-style fomula}\label{SecBS}

The price $F(x_t,t)$ is given by
\[F(x_t,t) = \varphi\left(
e^{-d(T-t)}S_t P_1(\varphi) - e^{-r(T-t)}K
  P_2(\varphi)\right)\]  and the hedge $\delta$ by
  \[\delta = \varphi e^{-d(T-t)} P_1(\varphi) \]

  where
$P_j(\varphi) = \frac{1-\varphi}{2} + \varphi \Pi_j$ for $j
\in\{1,2\}$
  with \\
$$\Pi_j = \frac{1}{2} +  \frac{1}{\pi} \int_0^\infty \Re\left[ \frac{\phi_j(k)}{i
  k}\right] dk$$
where  the characteristic functions $\phi_j$, for $j\in \{1,2\}$,
are given by: \[\dps \phi_j(k) = e^{i k X + A(k,\tau) + B(k,\tau)
V_0 + C(k,\tau) + D(k,\tau)\lambda}]\]

 Using the notations : \[ \left\{
\begin{array}{lll}
u = +1,\;\;  I=1, \; \; b = \kappa - \rho \sigma_v &if& j=1 \\
u = -1,\;\;  I=0, \; \; b = \kappa &if& j=2
\end{array}
\right. \] the coefficients $A(k,\tau)$, $B(k,\tau)$,
$C(k,\tau)$ and  $D(k,\tau)$  are given as follows: \\


\begin{enumerate}
\item Volatility: \\
\begin{itemize}
\item Constant volatility : \\
$A(k,\tau)=0$, $B(k,\tau)=-1/2(k^2-uik)\tau $. \item Stochastic
volatility (Heston) :\\

$\dps A(k,\tau)=-\frac{\kappa\theta}{\sigma_v^2}\left[\psi_+ \tau
+ 2ln\left( \frac{\psi_-+\psi_+e^{-\tau\zeta}}{2\zeta}\right) \right] $ \\
$\dps
B(k,\tau)=-(k^2-uik)\frac{1-e^{-\tau\zeta}}{\psi_-+\psi_+e^{-\tau\zeta}}
$

where $\dps\psi_\pm = \mp (b-\rho\sigma_vik) + \zeta$ and
$\dps\zeta=\sqrt{(b-\rho\sigma_vik)^2 + \sigma_v^2(k^2-iuk)}$.
\end{itemize}
%
\item Jumps :\\
\begin{itemize}
\item Merton model : constant jump rate intensity and log-normal jump size distribution \\
$C(k,\tau)=0$, $D(k,\tau)=\tau\Lambda(k)$ where \\
$\dps\Lambda(k) = e^{(m+Iv^2) ik - v^2 k^2 /2 + I(m+v^2/2)} -1 +
 (ik+I)(e^{m+v^2/2}-1)$.
\end{itemize}
\end{enumerate}
%
\section{Numerical integration}
%
In order to compute the infinite integrals, needed in the pricing
formulas, we use the approximation: 
\[ 
\int_0^\infty f(x)dx \simeq \sum_{j=0}^N \int_{jh}^{(j+1)h}f(x)dx.
\]
The number $N$ of the sub-integrals used is determinated when the
contribution of the last strip $[jh,(j+1)h]$ is smaller than a given
tolerance $\epsilon$. Each sub-integral $\dps\int_{jh}^{(j+1)h}f(x)dx$
is computed using a Gaussian quadrature.
%
\section{Implementation of the pricing routines}
\subsection{The {\tt svj.c} file}
This file contains the pricing routines, giving the price of an european
call or put in the Merton/Heston/Merton+Heston models. Any of the two pricing
formulas presented in Sections~\ref{SecCF} and~\ref{SecBS} can be used.
\subsection{The {\tt ft\_\emph{Opt}\_\emph{Model}.c} files}
%
\emph{Opt} is the option type, \emph{call} or \emph{put}. \emph{Model}
is the model used, \emph{merton} for the Merton model, \emph{heston} for
the Heston model and \emph{hestmert} for the combined model
Heston+Merton. 
In each file, we set the option type and the model parameters, next, we
call the {\tt calc\_price\_svj} rountine from {\tt svj.c } file. The
default pricing method used is the Black-Scholes like formula given in
\ref{SecBS}. 
%
\section{Computing the gradients}
%
For each model (Merton/Heston/Merton+Heston) we need to compute the
gradient of an option (call/put) price with respect of the different
parameters of the model. 

The parameters are: $V_0$, $\lambda$, $m$ and
$v$ for Merton model, $V_0$, $\kappa$, $\theta$, $\sigma_v$ and $\rho$
for Heston model, and $V_0$, $\kappa$, $\theta$, $\sigma_v$, $\rho$,
$V_0$, $\lambda$, $m$ and $v$ for Heston+Merton model. 

These gradients can be called from {\tt
  grad\_ft\_\emph{Opt}\_\emph{Model}.c} files
which sets the option type, and the model parameters then calls the
{\tt calc\_grad\_svj } routine from {\tt  grad\_svj.c }. This routine
analytically computes the gradient of the pricing formulas (Sections~\ref{SecCF}
and~\ref{SecBS})  with respect of each parameter of the given
model. The integration procedure is similaire to the one used from the
pricing formulas.  
%
\section{Calibration}
%
The goal is to find a parameter set $\alpha =(x_1,\dots,x_n)$ of a
model (Heston/Merton/Heston+Merton) fitting a given observed market
data (call/put prices or implied volatility surface). For an option
with a stike $K_i$ and a maturity $T_i$ we notice $P_i^{obs}$
(resp. $\sigma_i^{imp,obs}$) the observed option price (resp. implied
volatility) and $P_i$ (resp. $\sigma_i$) the model price (resp.
implied volatility). 
In order to measure the distance between the model and the market
prices, we define the following norms :
\begin{itemize}
\item prices norm: $\dps f^1_i = ||P_i^{obs} - P_i ||^2$
\item relative prices norm: $\dps f^2_i = ||\frac{P_i^{obs} - P_i}{P_i^{obs}}||^2$
\item implied volatility norm:  $\dps f^3_i = ||\sigma_i^{imp,obs} - \sigma_i||^2$
\end{itemize}  
%
The calibration is then the minimization of one the these norms for
the considered options: 
\[
\min_{x} f^j(x) = \min_{x} \sum_{i=1}^{N} w_i f^j_i
\]
where $w_i$ is a ponderation weight option $i$. 
%
%\subsection{General remarks}
The calibration strategies described in the next section are
developped and tested on synthetic data. Even if the case of real
market data is different, some conclusions on the behaviour of the
models are still valid, and the ideas proposed can be usefull for the
the elaboration of a robust calibration on real market data.

The synthetic market data are generated by pricing different option
of stikes $K_i$ and maturities $T_i$ with a model having a set of
parameters $x^\star = (x^\star_i,\dots,x^\star_n)$.  Then, starting 
with a random initial guess $x^0 = (x^0_i,\dots,x^0_n)$, the
calibration is considered as successful if it can find the parameter
$x^\star$. 

A first general remark is that the choice of the norm could be crucial
for the success of the calibration. In fact, the form of the surfaces we
are optimizing ($f^1$, $f^2$ or $f^3$) is different from a norm to
another: local minimum or flat surface in one case and convex in one
other. As expected, the norm $f^1$ is not appropriate as it gives very
different weigths for options out of the money or in the money for
example. The choice is then to alternate the norms $f^2$ and $f^3$. 

One model can have different sensitivities to each one of its
paramters. The Heston model, for example, is less sensitive to a
variation of $\kappa$  than to a variation of an other paramter.  
Some paramters can have a conjugate effects: a similtanious variation
of two parameters have less effects than a single variation. This is
the case for parameters $m_0$ and $v$ in the Merton model. 
On strategy proposed is to perfom search in subspace of the search
space: for example by locking, in the Heston model, all the parameters
expect $\kappa$ and then searching only in that direction. 

We notice that for Heston model and especially the Heston+Merton
model, we can find a parameter $x\neq x^\star$ where $f^2(x) \simeq
f^3(x) \simeq 0$. In other words, we can find two different set of
parameters giving very close (or even the same) smile surface. This
point is important for the stability of the calibration on real market
data.

%
%\subsection{Calibration strategies}
%strategie par model
%
\begin{thebibliography}{1}
\bibitem{sepp}
Artur Sepp.
\newblock {\em Pricing European-Style Options under Jump Diffusion
  Processes with Stochastic Volatility: Applications of Fourier
  Transform,}
\newblock Proceedings of the 7th Tartu Conference on Multivariate
  Statistics, (2004).
\newblock http://www.hot.ee/seppar/papers/stochjumpvols.pdf
\end{thebibliography}
%
\end{document}
