\documentclass[a4paper,11pt]{article}
\usepackage{amssymb,amsmath}
\begin{document}
\pagestyle{empty}
\begin{center}
{\Large Calibration by a weighted Monte Carlo method. }

\



\

\Large Routine documentation.

\

\Large Yael BEER-GABEL

\Large June 2003
\end{center}

\

\

\begin{flushleft}

\
This documentation is devoted to the weighted Monte-Carlo calibration routine
{\bf Interfaceutilisateur.c}. 





\section*{Functions}


\textbullet{``repartition  ''}: returns an approximation of the
cumulative distribution function of the Normal law
$$N(d)=\frac{1}{\sqrt{2\pi}}\int_{-\infty}^{d}e^{-x^{2}}\,dx$$
\vspace{0.5 cm}
\textbullet{``samplepathsgeneration''} : generates discretized
sample-paths
$$\omega^{(i)}=(\overline{S}_{\Delta
t}(\omega^{(i)}),\hdots,\overline{S}_{M \Delta t}(\omega^{(i)})) ,
i=1,\hdots,\nu$$ of the stochastic volatility model
\begin{equation*}
\begin{cases}
\overline{S}_{(n+1)\Delta t} =  \overline{S}_{n\Delta
t}(1+(r-q)\Delta t +\overline{\sigma}_{n\Delta
  t}\sqrt{\Delta t}g^1_{n+1}) , n=1,\hdots,M\\
\overline{\sigma}_{(n+1)\Delta t}= \sigma_0+
(\overline{\sigma}_{n\Delta t}-\sigma_0)(1- \beta \Delta t)+
\alpha \sqrt{\Delta t}(\rho g^1_{n+1}+\sqrt{1-\rho^2} g^2_{n+1})
\end{cases}
\end{equation*}by the standard Euler scheme with
timestep $\Delta t=\frac {T}{M}$.

\vspace{0.5 cm}

\textbullet{``payoffs ''}: for $ 1 \leq j \leq N$, this function evaluates
the actualized payoff of the jth calibration put with strike $K_j$
and maturity $T_j$ on each path : $$ \mbox{for }1\leq i\leq \nu,\;g_{ij}=e^{-rT_j}\left(K_{j}-
\overline{S}_{\big[\frac{T_j}{\Delta t}\big]\Delta t}^{i}\right)^{+}.$$ Then, it
computes $\alpha_j=\sup_{1 \leq i
  \leq \nu}|g_{ij}-P_j|$ where $P_j$ is the market price of the jth calibration Put and returns the normalized payoffs
  $\tilde{g}_{ij}=\frac{g_{ij}-P_j}{\alpha_j}$ for $ 1 \leq i \leq \nu$ and $1\leq j \leq N$.

\vspace{0.5 cm}

\textbullet{``functionW''}: computes $W(\tilde{\lambda}_1, \hdots,
\tilde{\lambda}_N)=\ln\bigg(\sum_{i=1}^{\nu}{e^{\sum_{j=1}^{N}{\tilde{\lambda}_j\tilde{g}_{ij}}}}\bigg)+\frac{1}{2}\sum_{j=1}^{N}{
  \frac{w_j}{\alpha_j^2}\tilde{\lambda}_j^2}$.

\vspace{0.5 cm}

\textbullet{``gradient''}:returns $(\tilde{\lambda}_{1}, \hdots,
\tilde{\lambda}_N)$ which minimizes  $W(\tilde{\lambda}_1, \hdots,
\tilde{\lambda}_N)$  by a conjugate gradient algorithm combined
with Wolfe's Rule.

\vspace{0.5 cm}

\textbullet{''weights''}: computes the optimal weights
$p_i=p(\omega^{(i)})=\frac{e^{\sum_{j=1}^{N}{\tilde{\lambda}_j\tilde{g}_{ij}}}}{\sum_{k=1}^{\nu}{e^{\sum_{j=1}^{N}{\tilde{\lambda}_j\tilde{g}_{kj}}}}},
1 \leq i \leq \nu$.

\section*{Input files}

There are 2 input files called \emph{name in parameters} and
\emph{name in data}.

\

\emph{name in parameters} contains the parameters such as the
\textbf{spot}, the interest rate \textbf{r}, the dividend rate
\textbf{q}, \textbf{option volatility} to choose whether you want
to enter the parameters of the a priori model of the sample paths
or if you prefer the routine to determine them, \textbf{option
step} to choose whether you want to enter the timestep for the
discretized Euler scheme or if you prefer the routine to fix it
, the maturity \textbf{horizon} ($T$), the mean volatility of the
generated paths \textbf{sigma0} ($\sigma_0$), the timestep \textbf{deltat} ($\Delta t$), the
number of calibration puts \textbf{N} , the perturbation coefficient of
the volatility
\textbf{alpha} ($\alpha$), the return to mean coefficient \textbf{beta}
($\beta$), the correlation coefficient
\textbf{rho} ($\rho$), \textbf{option constraint} to choose whether you want
to use penalized constraints ($w_j>0$) or not ($w_j=0$). The name of the parameter you
need to enter is written on the right end of each line.

\

\emph{name in data} contains the calibration datas: each line of
the file consists in the datas of each calibration put $ K_j$,$T_j$
and $P_j$, $j=1 \hdots N$.

\

Examples are given in those files.
\end{flushleft}
\end{document}
