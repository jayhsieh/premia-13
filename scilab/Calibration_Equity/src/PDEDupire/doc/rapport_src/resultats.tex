\section{R\'esultats num\'eriques de calibration}
\label{SEC:RESULTATS}

\subsection{Tests sur donn\'ees synth\'etiques~: choix de 
la grille $n \times m$ et du coefficient de r\'egularisation 
$\lambda$}

Dans un premier temps, nous testons l'algorithme de calibration 
sur donn\'ees synth\'etiques, c'est-\`a-dire sur des donn\'ees 
g\'en\'er\'ees avec notre mod\`ele en r\'esolvant 
l'\'equation de Dupire. Nous appelons ``$\sigma_{true}$'' 
la ``vraie'' volatilit\'e que nous utilisons pour 
cr\'eer les donn\'ees.
Ces donn\'ees correspondent \`a des call europ\'eens avec~:
\begin{itemize}
\item
$S_0 = 100$~;
\item
$r = 0.05$~;
\item
$q = 0.02$.
\end{itemize}
Concernant la r\'esolution num\'erique de l'EDP de Dupire, 
nous choisissons les param\`etres suivants~:
\begin{itemize}
\item
$\left[y_{min};y_{max}\right] = [-5.298;5.298]$ 
(donc $\left[K_{min};K_{max}\right] = [0.005;200]$) est 
discr\'etis\'e par $N~=~400$ points (grille r\'eguli\`ere)~;
\item
$\left[t_0;T_{max}\right] = [0;1]$ est discr\'etis\'e 
par $M = 100$ points~;
\item
$\theta = 0.5$.
\end{itemize}
Nous avons g\'en\'er\'e $22$ donn\'ees avec la vraie volatilit\'e 
suivante~:
$$
\sigma_{true}(S,t) = \displaystyle\frac{15}{S}
$$
pour $11$ prix d'exercice et $2$ maturit\'es~:
\begin{itemize}
\item 
$K = [90,92,94,...,110]$~;
\item
$T = [0.5,1]$.
\end{itemize}
En ce qui concerne l'optimisation, nous effectuons $30$ it\'erations 
de l'algorithme de Quasi-Newton sans contraintes, en partant d'une 
volatilit\'e initiale constante~:
$$
\sigma_{init} = 0.15.
$$
Pour chaque test d'inversion, nous comparons la vraie volatilit\'e 
\`a la volatilit\'e estim\'ee avec notre algorithme de calibration. 
Nous montrons \'egalement l'erreur relative entre les deux 
volatilit\'es ainsi que la d\'ecroissance du crit\`ere au cours 
des it\'erations.

Les premiers tests (figures~\ref{FIG:CALIB2_NM1_0}, 
\ref{FIG:CALIB2_NM1_10} et \ref{FIG:CALIB2_NM1_100}) correspondent 
\`a une grille de param\'etrage de $\sigma$ tr\`es grossi\`ere 
avec~:
$$
n = m = 1
$$
(donc $16$ degr\'es de libert\'e) et \`a diff\'erentes valeurs de 
$\lambda$~: $0$, $10$ et $100$. On constate que le meilleur 
r\'esultat est obtenu avec $\lambda = 0$. Le fait d'ajouter un 
terme de r\'egularisation n'a rien am\'elior\'e sur 
cet exemple.

Avec $\lambda = 0$, nous choisissons de discr\'etiser $\sigma$ 
sur une grille plus grossi\`ere avec~:
$$
n = m = 3
$$
($36$ degr\'es de libert\'e). Le r\'esultat montr\'e \`a la 
figure~\ref{FIG:CALIB2_NM3_0} est 
moins bon que celui obtenu avec une grille $1 \times 1$~; c'est 
tr\`es net lorsque l'on observe l'erreur relative.

Compte tenu de ces premiers r\'esultats, nous choisirons dans la 
suite une grille tr\`es grossi\`ere ($n = m = 1$) et un 
coefficient de r\'egularisation $\lambda = 0$.

\pagebreak

% FIG:CALIB2_NM1_0
\begin{figure}[!htbp]

\begin{center}
\begin{minipage}{5.8cm}
\centerline{\includegraphics
[width=5.8cm,angle=0,draft=\draft] 
{./fig/calib2_sigmatrue}
} 
\end{minipage}
\hspace*{0.1cm}
\begin{minipage}{5.8cm}
\centerline{\includegraphics
[width=5.8cm,angle=0,draft=\draft] 
{./fig/calib2_sigmaest_nm1_0}
}
\end{minipage}
\end{center}

\begin{center}
\begin{minipage}{5.8cm}
(a) Vraie volatilit\'e.
\end{minipage}
\hspace*{0.1cm}
\begin{minipage}{5.8cm}
(b) Volatilit\'e estim\'ee.
\end{minipage}
\end{center}

\medskip

\begin{center}
\begin{minipage}{5.8cm}
\centerline{\includegraphics
[width=5.8cm,angle=0,draft=\draft] 
{./fig/calib2_errorrel_nm1_0}
}
\end{minipage}
\hspace*{0.1cm}
\begin{minipage}{5.8cm}
\centerline{\includegraphics
[width=5.8cm,angle=0,draft=\draft] 
{./fig/calib2_decfcout_nm1_0}
}
\end{minipage}
\end{center}

\begin{center}
\begin{minipage}{5.8cm}
(c) Erreur relative (en \%).
\end{minipage}
\hspace*{0.1cm}
\begin{minipage}{5.8cm}
(d) D\'ecroissance du crit\`ere.
\end{minipage}
\end{center}

\caption{Calibration avec $n = m = 1$ et $\lambda = 0$.}
\label{FIG:CALIB2_NM1_0}
\end{figure}

\pagebreak

% FIG:CALIB2_NM1_10
\begin{figure}[!htbp]

\begin{center}
\begin{minipage}{5.8cm}
\centerline{\includegraphics
[width=5.8cm,angle=0,draft=\draft] 
{./fig/calib2_sigmatrue}
} 
\end{minipage}
\hspace*{0.1cm}
\begin{minipage}{5.8cm}
\centerline{\includegraphics
[width=5.8cm,angle=0,draft=\draft] 
{./fig/calib2_sigmaest_nm1_10}
}
\end{minipage}
\end{center}

\begin{center}
\begin{minipage}{5.8cm}
(a) Vraie volatilit\'e.
\end{minipage}
\hspace*{0.1cm}
\begin{minipage}{5.8cm}
(b) Volatilit\'e estim\'ee.
\end{minipage}
\end{center}

\medskip

\begin{center}
\begin{minipage}{5.8cm}
\centerline{\includegraphics
[width=5.8cm,angle=0,draft=\draft] 
{./fig/calib2_errorrel_nm1_10}
}
\end{minipage}
\hspace*{0.1cm}
\begin{minipage}{5.8cm}
\centerline{\includegraphics
[width=5.8cm,angle=0,draft=\draft] 
{./fig/calib2_decfcout_nm1_10}
}
\end{minipage}
\end{center}

\begin{center}
\begin{minipage}{5.8cm}
(c) Erreur relative (en \%).
\end{minipage}
\hspace*{0.1cm}
\begin{minipage}{5.8cm}
(d) D\'ecroissance du crit\`ere.
\end{minipage}
\end{center}

\caption{Calibration avec $n = m = 1$ et $\lambda = 10$.}
\label{FIG:CALIB2_NM1_10}
\end{figure}

\pagebreak

% FIG:CALIB2_NM1_100
\begin{figure}[!htbp]

\begin{center}
\begin{minipage}{5.8cm}
\centerline{\includegraphics
[width=5.8cm,angle=0,draft=\draft] 
{./fig/calib2_sigmatrue}
} 
\end{minipage}
\hspace*{0.1cm}
\begin{minipage}{5.8cm}
\centerline{\includegraphics
[width=5.8cm,angle=0,draft=\draft] 
{./fig/calib2_sigmaest_nm1_100}
}
\end{minipage}
\end{center}

\begin{center}
\begin{minipage}{5.8cm}
(a) Vraie volatilit\'e.
\end{minipage}
\hspace*{0.1cm}
\begin{minipage}{5.8cm}
(b) Volatilit\'e estim\'ee.
\end{minipage}
\end{center}

\medskip

\begin{center}
\begin{minipage}{5.8cm}
\centerline{\includegraphics
[width=5.8cm,angle=0,draft=\draft] 
{./fig/calib2_errorrel_nm1_100}
}
\end{minipage}
\hspace*{0.1cm}
\begin{minipage}{5.8cm}
\centerline{\includegraphics
[width=5.8cm,angle=0,draft=\draft] 
{./fig/calib2_decfcout_nm1_100}
}
\end{minipage}
\end{center}

\begin{center}
\begin{minipage}{5.8cm}
(c) Erreur relative (en \%).
\end{minipage}
\hspace*{0.1cm}
\begin{minipage}{5.8cm}
(d) D\'ecroissance du crit\`ere.
\end{minipage}
\end{center}

\caption{Calibration avec $n = m = 1$ et $\lambda = 100$.}
\label{FIG:CALIB2_NM1_100}
\end{figure}

\pagebreak

% FIG:CALIB2_NM3_0
\begin{figure}[!htbp]

\begin{center}
\begin{minipage}{5.8cm}
\centerline{\includegraphics
[width=5.8cm,angle=0,draft=\draft] 
{./fig/calib2_sigmatrue}
} 
\end{minipage}
\hspace*{0.1cm}
\begin{minipage}{5.8cm}
\centerline{\includegraphics
[width=5.8cm,angle=0,draft=\draft] 
{./fig/calib2_sigmaest_nm3_0}
}
\end{minipage}
\end{center}

\begin{center}
\begin{minipage}{5.8cm}
(a) Vraie volatilit\'e.
\end{minipage}
\hspace*{0.1cm}
\begin{minipage}{5.8cm}
(b) Volatilit\'e estim\'ee.
\end{minipage}
\end{center}

\medskip

\begin{center}
\begin{minipage}{5.8cm}
\centerline{\includegraphics
[width=5.8cm,angle=0,draft=\draft] 
{./fig/calib2_errorrel_nm3_0}
}
\end{minipage}
\hspace*{0.1cm}
\begin{minipage}{5.8cm}
\centerline{\includegraphics
[width=5.8cm,angle=0,draft=\draft] 
{./fig/calib2_decfcout_nm3_0}
}
\end{minipage}
\end{center}

\begin{center}
\begin{minipage}{5.8cm}
(c) Erreur relative (en \%).
\end{minipage}
\hspace*{0.1cm}
\begin{minipage}{5.8cm}
(d) D\'ecroissance du crit\`ere.
\end{minipage}
\end{center}

\caption{Calibration avec $n = m = 3$ et $\lambda = 0$.}
\label{FIG:CALIB2_NM3_0}
\end{figure}

\pagebreak

\subsection{R\'egularisation par une approche multi\'echelle}
\label{SSEC:MULTIECHELLE}

Avec le choix d'une grille de param\'etrage de $\sigma$ tr\`es 
grossi\`ere ($n = m = 1$), nous retrouvons une volatilit\'e 
tr\`es r\'eguli\`ere par construction puisque l'on ne peut 
pas repr\'esenter des volatilit\'es tr\`es irr\'eguli\`eres. 
La contre-partie est qu'il est difficile d'affiner $\sigma$ 
dans certaines r\'egions du domaine, ce qui permettrait 
de mieux coller aux donn\'ees.

Pour r\'esoudre ce probl\`eme, nous choisissons d'utiliser 
une approche multi\'echelle. Cette approche a d\'ej\`a utilis\'ee 
sur des probl\`emes inverses issus de la G\'eophysique~; voir par 
exemple Chavent {\em et~al.}~\cite{bun:geo:95} ou 
Cl\'ement~\cite{fcl:inria:99}. Le principe de la m\'ethode 
est de rechercher dans un premier temps l'allure g\'en\'erale 
de la volatilit\'e sur une grille tr\`es grossi\`ere, puis 
d'optimiser sur une grille plus fine en partant du point 
initial que l'on vient d'obtenir, et ainsi de suite. 
Par exemple~:
$$
\sigma_{init}^{1 \times 1} \longrightarrow 
{\rm (Q.-Newton)} \longrightarrow 
\sigma_{est}^{1 \times 1} \rightarrow 
\sigma_{init}^{3 \times 3} \longrightarrow 
{\rm (Q.-Newton)} \longrightarrow 
\sigma_{est}^{3 \times 3} \rightarrow ...
$$

\begin{center}
\begin{minipage}{4.5cm}
\centerline{\includegraphics[width=4.5cm,angle=0,draft=\draft]
{./fig/grille1}}
\end{minipage}
$\longrightarrow$
\begin{minipage}{4.5cm}
\centerline{\includegraphics[width=4.5cm,angle=0,draft=\draft]
{./fig/grille3}} 
\end{minipage}
$\longrightarrow$ ...
\end{center}

Nous allons appliquer cette m\'ethode sur les prochains 
r\'esultats sur donn\'ees synth\'etiques et sur donn\'ees 
r\'eelles.

\subsection{Tests sur donn\'ees synth\'etiques~: influence du 
point initial, choix d'un algorithme de Quasi-Newton avec bornes}
\label{SSEC:CALIB3}

Nous avons g\'en\'er\'e de nouvelles donn\'ees synth\'etiques 
correspondant \`a des call europ\'eens avec~:
\begin{itemize}
\item
$S_0 = 100$~;
\item
$r = 0.05$~;
\item
$q = 0.02$, 
\end{itemize}
avec le m\^eme choix de grille de diff\'erences finies que dans 
l'exemple pr\'ec\'edent~:
\begin{itemize}
\item
$\left[y_{min};y_{max}\right] = [-5.298;5.298]$ 
(donc $\left[K_{min};K_{max}\right] = [0.005;200]$) est 
discr\'etis\'e par $N~=~400$ points (grille r\'eguli\`ere)~;
\item
$\left[t_0;T_{max}\right] = [0;1]$ est discr\'etis\'e 
par $M = 100$ points~;
\item
$\theta = 0.5$.
\end{itemize}
La vraie volatilit\'e utilis\'ee est~:
$$
\sigma_{true}(S,t) = 0.05 
+ 0.1 \exp \left( -\frac{S}{100} \right) + 0.5 t
$$
et nous avons simul\'e $22$ donn\'ees correspondant aux couples 
$(K,T)$ suivants~:
\begin{itemize}
\item
$K = [90,92,94,...,110]$~;
\item
$T = [0.5,1]$.
\end{itemize}

Nous avons utilis\'e l'approche multi\'echelle d\'ecrite au 
paragraphe pr\'ec\'edent, en passant d'une grille $1 \times 1$ 
($30$ it\'erations) \`a une grille $3 \times 3$ 
($30$ it\'erations).

Nous avons test\'e la robustesse de l'algorithme en partant de 
diff\'erentes volatilit\'es initiales (constantes)~: 
$$
\sigma_{init} = 0.1,\;0.35,\;,0.5\;{\rm et}\;0.9.
$$
Les r\'esultats obtenus apr\`es utilisation de l'approche 
multi\'echelle (sur la grille $3 \times 3$) sont montr\'es 
respectivement aux figures~\ref{FIG:CALIB3_NM3_01}, 
\ref{FIG:CALIB3_NM3_035}, \ref{FIG:CALIB3_NM3_05} et 
\ref{FIG:CALIB3_NM3_09}. Notons que dans chaque cas, la 
volatilit\'e estim\'ee sur la grille $3 \times 3$ (apr\`es 
multi\'echelle) est proche (graphiquement) de celle 
estim\'ee sur la grille $1 \times 1$, c'est pourquoi nous ne 
montrons pas cette derni\`ere. En revanche, l'approche 
multi\'echelle a permis de faire d\'ecro\^{\i}tre encore la 
fonction co\^ut et ainsi de mieux reproduire les donn\'ees. 
Les trois premiers r\'esultats obtenus sont bons puisque 
l'erreur relative entre la vraie volatilit\'e et la 
volatilit\'e estim\'ee est inf\'erieure \`a $5$\% sur une 
grande partie du domaine. Si on part du point initial tr\`es 
\'eloign\'e $\sigma_{init} = 0.9$ (figure~ \ref{FIG:CALIB3_NM3_09}), 
on constate que la volatilit\'e estim\'ee poss\`ede des valeurs 
n\'egatives pour des maturit\'es inf\'erieures \`a $0.5$. Notons que 
ce n'est pas tr\`es g\^enant puisque c'est $\sigma^2$ qui intervient 
dans les calculs et non $\sigma$. Cependant, il est possible de 
r\'esoudre ce probl\`eme en ajoutant des contraintes de bornes sur 
$\sigma$ dans le probl\`eme d'optimisation.

Par cons\'equent, nous avons choisi de remplacer l'algorithme 
de Quasi-Newton sans contraintes utilis\'e jusqu'\`a pr\'esent 
par un algorithme de Quasi-Newton (\`a m\'emoire limit\'ee) avec 
contraintes de bornes. Il s'agit d'un algorithme d\'evelopp\'e 
par Nocedal {\em et~al.} (voir~\cite{noce:siam:95,noce:acm:97}). 
Notons que les contraintes de bornes sont impos\'ees sur les 
degr\'es de libert\'e de $\sigma$ correspondant aux valeurs 
de la fonction en chaque n{\oe}ud. Nous pouvons remarquer 
que ceci ne garantit pas que la fonction $\sigma$ n'ait pas de 
valeurs n\'egatives \`a l'int\'erieur du domaine puisque les 
degr\'es de libert\'e correpondant aux d\'eriv\'ees sont laiss\'es 
libres. 

En choisissant les bornes min et max \`a $0$ et \`a $1$, nous 
obtenons le r\'esultat montr\'e \`a la 
figure~\ref{FIG:CALIB3_V2_0_1_NM3_09}. La volatilit\'e estim\'ee 
ne contient plus de valeurs n\'egatives et l'erreur est inf\'erieure 
\`a $5$\% sur une grande partie du domaine. Nous avons 
\'egalement refait les tests de calibration en 
partant de $0.1$, $0.35$ et $0.5$ avec l'algorithme de Quasi-Newton 
avec bornes et les r\'esultats sont similaires \`a ceux obtenus 
pr\'ec\'edemment. Ainsi, en choisissant l'algorithme avec bornes, 
nous obtenons des volatilit\'es estim\'ees tr\`es satisfaisantes 
quel que soit le point initial choisi ($0.1$, $0.35$, $0.5$ ou 
$0.9$).

\pagebreak

% FIG:CALIB3_NM3_01
\begin{figure}[!htbp]

\begin{center}
\begin{minipage}{5.8cm}
\centerline{\includegraphics
[width=5.8cm,angle=0,draft=\draft] 
{./fig/calib3_sigmatrue}
} 
\end{minipage}
\hspace*{0.1cm}
\begin{minipage}{5.8cm}
\centerline{\includegraphics
[width=5.8cm,angle=0,draft=\draft] 
{./fig/calib3_sigmaest_nm3_01}
}
\end{minipage}
\end{center}

\begin{center}
\begin{minipage}{5.8cm}
(a) Vraie volatilit\'e.
\end{minipage}
\hspace*{0.1cm}
\begin{minipage}{5.8cm}
(b) Volatilit\'e estim\'ee.
\end{minipage}
\end{center}

\medskip

\begin{center}
\begin{minipage}{5.8cm}
\centerline{\includegraphics
[width=5.8cm,angle=0,draft=\draft] 
{./fig/calib3_errorrel_nm3_01}
}
\end{minipage}
\end{center}

\begin{center}
\begin{minipage}{5.8cm}
(c) Erreur relative (en \%).
\end{minipage}
\end{center}

\medskip

\begin{center}
\begin{minipage}{5.8cm}
\centerline{\includegraphics
[width=5.8cm,angle=0,draft=\draft] 
{./fig/calib3_decfcout_n1m1_0_01}
}
\end{minipage}
\hspace*{0.1cm}
\begin{minipage}{5.8cm}
\centerline{\includegraphics
[width=5.8cm,angle=0,draft=\draft] 
{./fig/calib3_decfcout_n3m3_0_01}
}
\end{minipage}
\end{center}

\begin{center}
\begin{minipage}{5.8cm}
(d) D\'ecroissance du crit\`ere~: $1 \times 1$.
\end{minipage}
\hspace*{0.1cm}
\begin{minipage}{5.8cm}
(e) D\'ecroissance du crit\`ere~: $3 \times 3$.
\end{minipage}
\end{center}

\caption{Calibration en partant de $\sigma_{init} = 0.1$.}
\label{FIG:CALIB3_NM3_01}
\end{figure}

\pagebreak

% FIG:CALIB3_NM3_035
\begin{figure}[!htbp]

\begin{center}
\begin{minipage}{5.8cm}
\centerline{\includegraphics
[width=5.8cm,angle=0,draft=\draft] 
{./fig/calib3_sigmatrue}
} 
\end{minipage}
\hspace*{0.1cm}
\begin{minipage}{5.8cm}
\centerline{\includegraphics
[width=5.8cm,angle=0,draft=\draft] 
{./fig/calib3_sigmaest_nm3_035}
}
\end{minipage}
\end{center}

\begin{center}
\begin{minipage}{5.8cm}
(a) Vraie volatilit\'e.
\end{minipage}
\hspace*{0.1cm}
\begin{minipage}{5.8cm}
(b) Volatilit\'e estim\'ee.
\end{minipage}
\end{center}

\medskip

\begin{center}
\begin{minipage}{5.8cm}
\centerline{\includegraphics
[width=5.8cm,angle=0,draft=\draft] 
{./fig/calib3_errorrel_nm3_035}
}
\end{minipage}
\end{center}

\begin{center}
\begin{minipage}{5.8cm}
(c) Erreur relative (en \%).
\end{minipage}
\end{center}

\medskip

\begin{center}
\begin{minipage}{5.8cm}
\centerline{\includegraphics
[width=5.8cm,angle=0,draft=\draft] 
{./fig/calib3_decfcout_n1m1_0_035}
}
\end{minipage}
\hspace*{0.1cm}
\begin{minipage}{5.8cm}
\centerline{\includegraphics
[width=5.8cm,angle=0,draft=\draft] 
{./fig/calib3_decfcout_n3m3_0_035}
}
\end{minipage}
\end{center}

\begin{center}
\begin{minipage}{5.8cm}
(d) D\'ecroissance du crit\`ere~: $1 \times 1$.
\end{minipage}
\hspace*{0.1cm}
\begin{minipage}{5.8cm}
(e) D\'ecroissance du crit\`ere~: $3 \times 3$.
\end{minipage}
\end{center}

\caption{Calibration en partant de $\sigma_{init} = 0.35$.}
\label{FIG:CALIB3_NM3_035}
\end{figure}

\pagebreak

% FIG:CALIB3_NM3_05
\begin{figure}[!htbp]

\begin{center}
\begin{minipage}{5.8cm}
\centerline{\includegraphics
[width=5.8cm,angle=0,draft=\draft] 
{./fig/calib3_sigmatrue}
} 
\end{minipage}
\hspace*{0.1cm}
\begin{minipage}{5.8cm}
\centerline{\includegraphics
[width=5.8cm,angle=0,draft=\draft] 
{./fig/calib3_sigmaest_nm3_05}
}
\end{minipage}
\end{center}

\begin{center}
\begin{minipage}{5.8cm}
(a) Vraie volatilit\'e.
\end{minipage}
\hspace*{0.1cm}
\begin{minipage}{5.8cm}
(b) Volatilit\'e estim\'ee.
\end{minipage}
\end{center}

\medskip

\begin{center}
\begin{minipage}{5.8cm}
\centerline{\includegraphics
[width=5.8cm,angle=0,draft=\draft] 
{./fig/calib3_errorrel_nm3_05}
}
\end{minipage}
\end{center}

\begin{center}
\begin{minipage}{5.8cm}
(c) Erreur relative (en \%).
\end{minipage}
\end{center}

\medskip

\begin{center}
\begin{minipage}{5.8cm}
\centerline{\includegraphics
[width=5.8cm,angle=0,draft=\draft] 
{./fig/calib3_decfcout_n1m1_0_05}
}
\end{minipage}
\hspace*{0.1cm}
\begin{minipage}{5.8cm}
\centerline{\includegraphics
[width=5.8cm,angle=0,draft=\draft] 
{./fig/calib3_decfcout_n3m3_0_05}
}
\end{minipage}
\end{center}

\begin{center}
\begin{minipage}{5.8cm}
(d) D\'ecroissance du crit\`ere~: $1 \times 1$.
\end{minipage}
\hspace*{0.1cm}
\begin{minipage}{5.8cm}
(e) D\'ecroissance du crit\`ere~: $3 \times 3$.
\end{minipage}
\end{center}

\caption{Calibration en partant de $\sigma_{init} = 0.5$.}
\label{FIG:CALIB3_NM3_05}
\end{figure}

\pagebreak

% FIG:CALIB3_NM3_09
\begin{figure}[!htbp]

\begin{center}
\begin{minipage}{5.8cm}
\centerline{\includegraphics
[width=5.8cm,angle=0,draft=\draft] 
{./fig/calib3_sigmatrue}
} 
\end{minipage}
\hspace*{0.1cm}
\begin{minipage}{5.8cm}
\centerline{\includegraphics
[width=5.8cm,angle=0,draft=\draft] 
{./fig/calib3_sigmaest_nm3_09}
}
\end{minipage}
\end{center}

\begin{center}
\begin{minipage}{5.8cm}
(a) Vraie volatilit\'e.
\end{minipage}
\hspace*{0.1cm}
\begin{minipage}{5.8cm}
(b) Volatilit\'e estim\'ee.
\end{minipage}
\end{center}

\medskip

\begin{center}
\begin{minipage}{5.8cm}
\centerline{\includegraphics
[width=5.8cm,angle=0,draft=\draft] 
{./fig/calib3_errorrel_nm3_09}
}
\end{minipage}
\end{center}

\begin{center}
\begin{minipage}{5.8cm}
(c) Erreur relative (en \%).
\end{minipage}
\end{center}

\medskip

\begin{center}
\begin{minipage}{5.8cm}
\centerline{\includegraphics
[width=5.8cm,angle=0,draft=\draft] 
{./fig/calib3_decfcout_n1m1_0_09}
}
\end{minipage}
\hspace*{0.1cm}
\begin{minipage}{5.8cm}
\centerline{\includegraphics
[width=5.8cm,angle=0,draft=\draft] 
{./fig/calib3_decfcout_n3m3_0_09}
}
\end{minipage}
\end{center}

\begin{center}
\begin{minipage}{5.8cm}
(d) D\'ecroissance du crit\`ere~: $1 \times 1$.
\end{minipage}
\hspace*{0.1cm}
\begin{minipage}{5.8cm}
(e) D\'ecroissance du crit\`ere~: $3 \times 3$.
\end{minipage}
\end{center}

\caption{Calibration en partant de $\sigma_{init} = 0.9$.}
\label{FIG:CALIB3_NM3_09}
\end{figure}

\pagebreak

% FIG:CALIB3_V2_0_1_NM3_09
\begin{figure}[!htbp]

\begin{center}
\begin{minipage}{5.8cm}
\centerline{\includegraphics
[width=5.8cm,angle=0,draft=\draft] 
{./fig/calib3_sigmatrue}
} 
\end{minipage}
\hspace*{0.1cm}
\begin{minipage}{5.8cm}
\centerline{\includegraphics
[width=5.8cm,angle=0,draft=\draft] 
{./fig/calib3_v2_0_1_sigmaest_nm3_09}
}
\end{minipage}
\end{center}

\begin{center}
\begin{minipage}{5.8cm}
(a) Vraie volatilit\'e.
\end{minipage}
\hspace*{0.1cm}
\begin{minipage}{5.8cm}
(b) Volatilit\'e estim\'ee.
\end{minipage}
\end{center}

\medskip

\begin{center}
\begin{minipage}{5.8cm}
\centerline{\includegraphics
[width=5.8cm,angle=0,draft=\draft] 
{./fig/calib3_v2_0_1_errorrel_nm3_09}
}
\end{minipage}
\end{center}

\begin{center}
\begin{minipage}{5.8cm}
(c) Erreur relative (en \%).
\end{minipage}
\end{center}

\medskip

\begin{center}
\begin{minipage}{5.8cm}
\centerline{\includegraphics
[width=5.8cm,angle=0,draft=\draft] 
{./fig/calib3_v2_0_1_decfcout_n1m1_0_09}
}
\end{minipage}
\hspace*{0.1cm}
\begin{minipage}{5.8cm}
\centerline{\includegraphics
[width=5.8cm,angle=0,draft=\draft] 
{./fig/calib3_v2_0_1_decfcout_n3m3_0_09}
}
\end{minipage}
\end{center}

\begin{center}
\begin{minipage}{5.8cm}
(d) D\'ecroissance du crit\`ere~: $1 \times 1$.
\end{minipage}
\hspace*{0.1cm}
\begin{minipage}{5.8cm}
(e) D\'ecroissance du crit\`ere~: $3 \times 3$.
\end{minipage}
\end{center}

\caption{Calibration en partant de $\sigma_{init} = 0.9$ avec un 
algorithme de Quasi-Newton avec bornes (0-1).}
\label{FIG:CALIB3_V2_0_1_NM3_09}
\end{figure}

\clearpage
\pagebreak

\subsection{Tests sur donn\'ees r\'eelles du march\'e}

Nous avons ensuite test\'e notre algorithme de calibration 
sur des donn\'ees r\'eelles r\'ecup\'er\'ees dans l'article 
de Coleman~{\em et~al.}~\cite{col:jcf:99}. Il s'agit de $70$ call 
europ\'eens provenant du ``European S\&P 500 index'' avec~:
\begin{itemize}
\item
$S_0 = 590$~;
\item
$r = 0.06$~;
\item
$q = 0.0262$~;
\item
$10$ prix d'exercice~: 
$K = [501.5,531,560.5,590,619.5,649,678.5,708,767,826]$~;
\item
$7$ maturit\'es~: $T = [0.175,0.425,0.695,0.94,1,1.5,2]$.
\end{itemize}

Nous avons choisi $\lambda = 0$ et le point initial suivant~:
$$
\sigma_{init} = 0.13
$$
correspondant \`a la moyenne des volatilit\'es implicites. 
Nous comparons les r\'esultats obtenus~:
\begin{itemize}
\item
sans l'approche multi\'echelle~: $120$ it\'erations sur une 
grille $1 \times 1$~;
\item
avec l'approche multi\'echelle~: $4 \times 30$ it\'erations 
sur des grilles $1 \times 1$, $3 \times 3$, $6 \times 6$ puis 
$12 \times 12$.
\end{itemize}

Nous avons utilis\'e dans un premier temps l'algorithme de 
Quasi-Newton sans contraintes. Les r\'esultats obtenus sans et 
avec l'approche multi\'echelle sont montr\'es respectivement aux 
figures~\ref{FIG:REEL1_SANS} et~\ref{FIG:REEL1_AVEC}. On constate 
qu'avec l'approche multi\'echelle, la fonction co\^ut descend plus 
bas (cf. figure~\ref{FIG:REEL1_DECFCOUT} \`a gauche), donc on 
reproduit mieux les donn\'ees avec cette volatilit\'e estim\'ee. 
On constate qu'entre la grille $1 \times 1$ et la grille 
$12 \times 12$, l'allure g\'en\'erale est conserv\'ee et la 
volatilit\'e estim\'ee sur la grille $12 \times 12$ comporte de 
l\'eg\`eres variations permettant de mieux expliquer les donn\'ees 
dans certaines zones. Cependant, il y a des valeurs n\'egatives 
dans une petite partie du domaine o\`u la volatilit\'e influe 
peu sur les donn\'ees.

Pour r\'esoudre ce probl\`eme, nous utilisons \`a nouveau 
l'algorithme de Quasi-Newton avec bornes. En fixant les 
bornes min et max \`a $0$ et \`a $1$, nous obtenons encore des 
valeurs n\'egatives dans certaines parties du domaine. Cependant, 
en augmentant la borne min \`a $0.08$, la volatilit\'e estim\'ee 
ne contient plus de valeurs n\'egatives, comme le montrent les 
r\'esultats obtenus sans et avec l'approche multi\'echelle 
(figures~\ref{FIG:REEL1_SANS_V2_008_1} 
et~\ref{FIG:REEL1_AVEC_V2_008_1}). A nouveau, l'approche 
multi\'echelle permet de faire d\'ecro\^{\i}tre la fonction 
co\^ut vers une valeur plus basse (cf. 
figure~\ref{FIG:REEL1_DECFCOUT} \`a droite), proche de la valeur 
obtenue avec l'algorithme de Quasi-Newton sans contraintes ($1.69$ 
au lieu de $1.58$). La volatilit\'e estim\'ee apr\`es utilisation 
de l'approche multi\'echelle (sur la grille $12 \times 12$) 
poss\`ede des valeurs comprises entre $0.05$ et $0.3$ qui 
ne semblent pas aberrantes. Compar\'ee aux r\'esultats obtenus par 
Coleman~{\em et~al.}~\cite{col:jcf:99}, la volatilit\'e que nous 
avons estim\'ee poss\`ede des oscillations comparables entre $0.1$ et 
$0.25$, mais elle ne poss\`ede pas de grandes valeurs (proches 
de $0.7$) pour des maturit\'es proches de $0$.

\pagebreak

% FIG:REEL1_SANS
\begin{figure}[!htbp]

\begin{center}
\begin{minipage}{5.8cm}
\centerline{\includegraphics
[width=5.8cm,angle=0,draft=\draft] 
{./fig/reel1_sigmaest_n1m1_0_013}
} 
\end{minipage}
\hspace*{0.1cm}
\begin{minipage}{5.8cm}
\centerline{\includegraphics
[width=5.8cm,angle=0,draft=\draft] 
{./fig/reel1_sigmaest60_n1m1_0_013}
}
\end{minipage}
\end{center}

\begin{center}
\begin{minipage}{5.8cm}
(a) $30$ it\'erations.
\end{minipage}
\hspace*{0.1cm}
\begin{minipage}{5.8cm}
(b) $60$ it\'erations.
\end{minipage}
\end{center}

\medskip

\begin{center}
\begin{minipage}{5.8cm}
\centerline{\includegraphics
[width=5.8cm,angle=0,draft=\draft] 
{./fig/reel1_sigmaest90_n1m1_0_013}
}
\end{minipage}
\hspace*{0.1cm}
\begin{minipage}{5.8cm}
\centerline{\includegraphics
[width=5.8cm,angle=0,draft=\draft] 
{./fig/reel1_sigmaest120_n1m1_0_013}
}
\end{minipage}
\end{center}

\begin{center}
\begin{minipage}{5.8cm}
(c) $90$ it\'erations.
\end{minipage}
\hspace*{0.1cm}
\begin{minipage}{5.8cm}
(d) $120$ it\'erations.
\end{minipage}
\end{center}

\caption{Calibration sur donn\'ees r\'eelles avec $n = m = 1$ sans 
utiliser l'approche multi\'echelle. Algorithme de Quasi-Newton 
sans contraintes. \newline
$F : 263.5 \searrow 5.36 \searrow 4.63 
\searrow 3.27 \searrow 3.24$.
}
\label{FIG:REEL1_SANS}
\end{figure}

\pagebreak

% FIG:REEL1_AVEC
\begin{figure}[!htbp]

\begin{center}
\begin{minipage}{5.8cm}
\centerline{\includegraphics
[width=5.8cm,angle=0,draft=\draft] 
{./fig/reel1_sigmaest_n1m1_0_013}
} 
\end{minipage}
\hspace*{0.1cm}
\begin{minipage}{5.8cm}
\centerline{\includegraphics
[width=5.8cm,angle=0,draft=\draft] 
{./fig/reel1_sigmaest_n3m3_0_013}
}
\end{minipage}
\end{center}

\begin{center}
\begin{minipage}{5.8cm}
(a) $30$ it\'erations~: grille $1 \times 1$.
\end{minipage}
\hspace*{0.1cm}
\begin{minipage}{5.8cm}
(b) $30$ it\'erations~: grille $3 \times 3$.
\end{minipage}
\end{center}

\medskip

\begin{center}
\begin{minipage}{5.8cm}
\centerline{\includegraphics
[width=5.8cm,angle=0,draft=\draft] 
{./fig/reel1_sigmaest_n6m6_0_013}
}
\end{minipage}
\hspace*{0.1cm}
\begin{minipage}{5.8cm}
\centerline{\includegraphics
[width=5.8cm,angle=0,draft=\draft] 
{./fig/reel1_sigmaest_n12m12_0_013}
}
\end{minipage}
\end{center}

\begin{center}
\begin{minipage}{5.8cm}
(c) $30$ it\'erations~: grille $6 \times 6$.
\end{minipage}
\hspace*{0.1cm}
\begin{minipage}{5.8cm}
(d) $30$ it\'erations~: grille $12 \times 12$.
\end{minipage}
\end{center}

\caption{Calibration sur donn\'ees r\'eelles en utilisant 
l'approche multi\'echelle. Algorithme de Quasi-Newton 
sans contraintes. \newline
$F : 263.5 \searrow 5.36 \searrow 3.20 
\searrow 2.09 \searrow 1.58$.
}
\label{FIG:REEL1_AVEC}
\end{figure}

\pagebreak

% FIG:REEL1_SANS_V2_008_1
\begin{figure}[!htbp]

\begin{center}
\begin{minipage}{5.8cm}
\centerline{\includegraphics
[width=5.8cm,angle=0,draft=\draft] 
{./fig/reel1_v2_008_1_sigmaest_n1m1_0_013}
} 
\end{minipage}
\hspace*{0.1cm}
\begin{minipage}{5.8cm}
\centerline{\includegraphics
[width=5.8cm,angle=0,draft=\draft] 
{./fig/reel1_v2_008_1_sigmaest60_n1m1_0_013}
}
\end{minipage}
\end{center}

\begin{center}
\begin{minipage}{5.8cm}
(a) $30$ it\'erations.
\end{minipage}
\hspace*{0.1cm}
\begin{minipage}{5.8cm}
(b) $60$ it\'erations.
\end{minipage}
\end{center}

\medskip

\begin{center}
\begin{minipage}{5.8cm}
\centerline{\includegraphics
[width=5.8cm,angle=0,draft=\draft] 
{./fig/reel1_v2_008_1_sigmaest90_n1m1_0_013}
}
\end{minipage}
\hspace*{0.1cm}
\begin{minipage}{5.8cm}
\centerline{\includegraphics
[width=5.8cm,angle=0,draft=\draft] 
{./fig/reel1_v2_008_1_sigmaest120_n1m1_0_013}
}
\end{minipage}
\end{center}

\begin{center}
\begin{minipage}{5.8cm}
(c) $90$ it\'erations.
\end{minipage}
\hspace*{0.1cm}
\begin{minipage}{5.8cm}
(d) $120$ it\'erations.
\end{minipage}
\end{center}

\caption{Calibration sur donn\'ees r\'eelles avec $n = m = 1$ sans 
utiliser l'approche multi\'echelle. Algorithme de Quasi-Newton 
avec bornes (0.08-1). \newline
$F : 263.5 \searrow 90.47 \searrow 13.21 
\searrow 11.30 \searrow 10.95$.
}
\label{FIG:REEL1_SANS_V2_008_1}
\end{figure}

\pagebreak

% FIG:REEL1_AVEC_V2_008_1
\begin{figure}[!htbp]

\begin{center}
\begin{minipage}{5.8cm}
\centerline{\includegraphics
[width=5.8cm,angle=0,draft=\draft] 
{./fig/reel1_v2_008_1_sigmaest_n1m1_0_013}
} 
\end{minipage}
\hspace*{0.1cm}
\begin{minipage}{5.8cm}
\centerline{\includegraphics
[width=5.8cm,angle=0,draft=\draft] 
{./fig/reel1_v2_008_1_sigmaest_n3m3_0_013}
}
\end{minipage}
\end{center}

\begin{center}
\begin{minipage}{5.8cm}
(a) $30$ it\'erations~: grille $1 \times 1$.
\end{minipage}
\hspace*{0.1cm}
\begin{minipage}{5.8cm}
(b) $30$ it\'erations~: grille $3 \times 3$.
\end{minipage}
\end{center}

\medskip

\begin{center}
\begin{minipage}{5.8cm}
\centerline{\includegraphics
[width=5.8cm,angle=0,draft=\draft] 
{./fig/reel1_v2_008_1_sigmaest_n6m6_0_013}
}
\end{minipage}
\hspace*{0.1cm}
\begin{minipage}{5.8cm}
\centerline{\includegraphics
[width=5.8cm,angle=0,draft=\draft] 
{./fig/reel1_v2_008_1_sigmaest_n12m12_0_013}
}
\end{minipage}
\end{center}

\begin{center}
\begin{minipage}{5.8cm}
(c) $30$ it\'erations~: grille $6 \times 6$.
\end{minipage}
\hspace*{0.1cm}
\begin{minipage}{5.8cm}
(d) $30$ it\'erations~: grille $12 \times 12$.
\end{minipage}
\end{center}

\caption{Calibration sur donn\'ees r\'eelles en utilisant 
l'approche multi\'echelle. Algorithme de Quasi-Newton 
avec bornes (0.08-1). \newline
$F : 263.5 \searrow 90.47 \searrow 10.06 
\searrow 3.66 \searrow 1.69$.
}
\label{FIG:REEL1_AVEC_V2_008_1}
\end{figure}

\pagebreak

% FIG:REEL1_DECFCOUT
\begin{figure}[!htbp]

\begin{center}
\begin{minipage}{5.8cm}
\centerline{\includegraphics
[width=5.8cm,angle=0,draft=\draft] 
{./fig/reel1_decfcout}
}
\end{minipage}
\hspace*{0.1cm}
\begin{minipage}{5.8cm}
\centerline{\includegraphics
[width=5.8cm,angle=0,draft=\draft] 
{./fig/reel1_v2_008_1_decfcout}
}
\end{minipage}
\end{center}

\caption{D\'ecroissance du crit\`ere avec ou sans l'approche 
multi\'echelle. A gauche~: Quasi-Newton sans contraintes. 
A droite~: Quasi-Newton avec bornes.
}
\label{FIG:REEL1_DECFCOUT}
\end{figure}
