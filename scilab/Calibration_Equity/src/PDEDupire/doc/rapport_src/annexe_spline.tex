\section{Interpolation par splines bicubiques}
\label{ANN:SPLINES}

Nous montrons dans cette annexe comment d\'eterminer la 
valeur d'une spline bicubique en un point quelconque du domaine 
\`a partir de ses degr\'es de libert\'e. Le d\'etail des 
d\'emonstrations se trouve dans l'article \cite{deb:jmp:62}. 
On consid\`ere une grille rectangulaire de points $(x_i,y_j)$, 
$i=0,...,I$, $j=0,...,J$. Soient~: 
\begin{itemize}
\item $u_{ij} = u(x_i,y_j)$ donn\'es pour $i=0,...,I$ et 
$j=0,...,J$,
\item $p_{ij} = u_x(x_i,y_j)$ donn\'es pour $i=0,I$ et 
$j=0,...,J$,
\item $q_{ij} = u_y(x_i,y_j)$ donn\'es pour $i=0,...,I$ et $j=0,J$,
\item $s_{ij} = u_{xy}(x_i,y_j)$ donn\'es aux 4 coins de la grille, 
i.e. pour $i=0,I$ et $j=0,J$.
\end{itemize}
Le probl\`eme consiste \`a interpoler ces points \`a l'aide d'une 
fonction $u \in C^2$.

\subsection{Splines cubiques en dimension 1}

Pour les fonctions d'une variable, l'interpolation par spline 
d\'efinit une fonction $u(x)$ de classe $C^2$ \'etant donn\'ees~: 
\begin{itemize}
\item les valeurs $u(x_i)$ aux points $x_i$, $i=0,...,I$, 
$x_0<...<x_I$,
\item les d\'eriv\'ees $p_i = u'(x_i)$ en $x_0$ et $x_I$.
\end{itemize}
La fonction d'interpolation est une fonction polynomiale cubique 
sur chacun des intervalles $[x_{i-1},x_i]$, $i=1,...,I$. 

Soient $\Delta x_i = x_{i+1}-x_i$ et $\Delta u_i = u_{i+1}-u_i$. 

\begin{theo}
Soit $u$ une fonction polynomiale cubique par morceaux de classe 
$C^1$. Pour des valeurs $u(x_i),~ i=1,...,I$ et $u'(x_0), u'(x_I)$ 
donn\'ees, il existe un unique ensemble de valeurs 
$p_i = u'(x_i), ~ i=1,...,I-1$ tel que $u \in C^2$. Ces valeurs 
sont d\'etermin\'ees par le syst\`eme d'\'equations suivant~:   
\begin{eqnarray}
\lefteqn{\Delta x_{i}p_{i-1}+2(\Delta x_{i}+\Delta x_{i-1})p_{i} + 
\Delta x_{i-1} p_{i+1} = } \nonumber \\
&& 3\biggl [ \frac{\Delta x_{i-1}}{\Delta x_i}\Delta u_i + 
\frac{\Delta x_i}{\Delta x_{i-1}}\Delta u_{i-1}\biggr ],~ i=1,...,I-1.
\end{eqnarray}
Par cons\'equent, pour chaque ensemble de donn\'ees 
$\{u_0,...,u_I,p_0,p_I\}$, il existe une unique fonction $u$ 
polynomiale cubique par morceaux de classe $C^1$ telle que~: 
\begin{xalignat*}{4} 
u(x_i)&=u_i, &\quad  i&=0,...,I,&\quad u'(x_0)&=p_0, 
&\quad  u'(x_I)&=p_I.
\end{xalignat*}
\end{theo}

\begin{corr}
$S(x;x_0,...,x_I)$ est un espace de dimension $(I+3)$.
\end{corr}

\begin{corr}
\label{base}
L'ensemble des fonctions $\Phi_i(x) \in S(x;x_0,...,x_I)$, 
$i=0,...,I+2$ d\'efinies par les conditions~: 
$$
\begin{array}{l}
\Phi_j(x_i) = \delta_{ij}  ~\text{pour} ~ i,j=0,...,I,\\
\Phi'_j(x_0) = \Phi'_j(x_I) = 0 ~ \text{pour} ~ j=0,...,I,\\
\Phi_{I+1}(x_i) = \Phi_{I+2}(x_i) = 0 ~ \text{pour} ~ i=0,...,I,\\
\Phi'_{I+1}(x_0) = \Phi'_{I+2}(x_I) = 1,\\
\Phi'_{I+1}(x_I) = \Phi'_{I+2}(x_0) = 0
\end{array}
$$
est une base de $S(x;x_0,...,x_I)$.
\end{corr}

\subsection{Splines bicubiques en dimension 2}

Soit $\{\psi_j(y)\}_{0\leq j\leq J+2}$ la base de $S(y;y_0,...,y_I)$ 
d\'efinie dans le corrolaire \ref{base} pr\'ec\'edent. Consid\'erons 
le produit tensoriel 
$T=S(x;x_0,...,x_I) \bigotimes S(y;y_0,...,y_J)$. $T$ est l'espace 
de dimension $(I+3)(J+3)$ des fonctions de la forme
\begin{eqnarray}
u(x,y) &=& 
\sum_{p=0}^{I+2}\sum_{q=0}^{J+2}\beta_{pq}\Phi_p(x)\Psi_q(y),
\end{eqnarray}
Par construction, $u$ est bicubique par morceaux et de classe $C^2$.

\begin{defi}
Soit $R_{ij} = \{(x,y):x_{i-1}\leq x \leq x_i, 
y_{j-1} \leq y \leq y_j\}$. Une fonction est polynomiale bicubique 
par morceaux si elle est d\'efinie sur chaque rectangle $R_{ij}$ de 
la grille par une fonction polynomiale bicubique, i.e.
$$
u(x,y) ~=~ c_{ij}(x,y) ~=~ \sum_{p=0}^3 \sum_{q=0}^3 
\alpha_{pq}^{ij}(x-x_{i-1})^p(y-y_{j-1})^q, ~ (x,y) \in R_{ij}.
$$
\end{defi}

\begin{theo}
Si les valeurs suivantes sont donn\'ees~: 
\begin{eqnarray}
&&\begin{cases}
u_{ij} = u(x_i,y_j), & i=0,...,I; j=0,...,J,\\
p_{ij} = u_x(x_i,y_j), & i=0,I; j=0,...,J,\\
q_{ij} = u_y(x_i,y_j), & i=0,...,I; j=0,J,\\
s_{ij} = u_{xy}(x_i,y_j), & i=0,I; j=0,J,\\
\end{cases}
\label{cond}
\end{eqnarray}
alors il existe une unique fonction bicubique par morceaux de classe 
$C^2$ qui v\'erifie~(\ref{cond}).
\end{theo}

\subsection{Evaluation de la fonction d'interpolation}

Dans cette section, nous pr\'esentons une m\'ethode d'\'evaluation 
de la fonction d'interpolation $u$ en un point $(x,y)$ quelconque. 
Rappelons que sur chaque rectangle $R_{ij}$, la fonction $u$ est 
\'egale \`a la fonction polynomiale bicubique 
\begin{eqnarray}
\label{Cij}
c_{ij} &=& \sum_{p=0}^3 \sum_{q=0}^3 
\gamma_{pq}^{ij}(x-x_{i-1})^p(y-y_{j-1})^q.
\end{eqnarray}

\begin{lemme}
Si $u_{ij}$, $p_{ij}$, $q_{ij}$ et $s_{ij}$ sont connus aux quatre 
coins du rectangle $R_{ij}$, alors il existe une unique fonction 
polynomiale bicubique $c_{ij}(x,y)$ qui concorde avec ces valeurs. La 
matrice $\Lambda_{ij}$ des coefficients $\gamma_{mn}^{ij}$ dans 
(\ref{Cij}) est donn\'ee par l'\'equation matricielle suivante~:
\begin{eqnarray}
\label{coeffs}
\Lambda_{ij} &=& A(\Delta x_{i-1}) K_{ij} A(\Delta y_{j-1})^T,
\end{eqnarray}
o\`u
\begin{eqnarray*}
K_{ij} = \begin{pmatrix} 
B_{i-1,j-1} & B_{i-1,j}\\ 
B_{i,j-1} & B_{ij} 
\end{pmatrix} & \text{avec} & B_{mn} = 
\begin{pmatrix} 
u_{mn}&q_{mn}\\p_{mn}&s_{mn}
\end{pmatrix},
\end{eqnarray*}
et o\`u la matrice $A(h)$ est d\'efinie par 
\begin{eqnarray*}
A(h) &=& \begin{pmatrix}
1&0&0&0\\
0&1&0&0\\
-3/h^2 & -2/h & 3/h^2 & -1/h \\
2/h^3 & 1/h^2 & -2/h^3 & 1/h^2 
\end{pmatrix}.
\end{eqnarray*}

\end{lemme}

\subsubsection*{D\'eriv\'ees aux points de la grille}

\begin{lemme}
Si les valeurs (\ref{cond}) sont connues, alors les valeurs
\begin{itemize}
\item $p_{ij}=u_x(x_i,y_j)$, pour $i=1,...,I-1$;$j=0,...,J$,
\item $q_{ij}=u_y(x_i,y_j)$, pour $i=0,...,I$;$j=1,...,J-1$,
\item $s_{ij}=u_{xy}(x_i,y_j)$, pour $i=1,...,I-1$;$j=0,J$ et 
$i=0,...,I$;$j=1,...,J-1$,
\end{itemize}
sont d\'etermin\'ees de mani\`ere unique par les $2I+J+5$ 
syst\`emes lin\'eaires qui comptent en tout $3IJ+I+J-5$ \'equations. 
Pour $j=0,...,J$, 
\begin{eqnarray}
\lefteqn{\Delta x_{i-1}p_{i+1,j}+2(\Delta x_{i-1}+\Delta x_i)p_{ij} 
+ \Delta x_i p_{i-1,j} = } \nonumber \\
&& 3\biggl [ \frac{\Delta x_{i-1}}{\Delta x_i}(u_{i+1,j}-u_{i,j}) 
+ \frac{\Delta x_i}{\Delta x_{i-1}}(u_{i,j}-u_{i-1,j})\biggr ],
~ i=1,...,I-1; \label{syst_spline1}
\end{eqnarray}
pour $i=0,...,I$,   
\begin{eqnarray}
\lefteqn{\Delta y_{j-1}q_{i,j+1}+2(\Delta y_{j-1}+\Delta y_j)q_{ij} 
+ \Delta y_j q_{i,j-1} = } \nonumber \\
&& 3\biggl [ \frac{\Delta y_{j-1}}{\Delta y_j}(u_{i,j+1}-u_{i,j}) 
+ \frac{\Delta y_j}{\Delta y_{j-1}}(u_{i,j}-u_{i,j-1})\biggr ],
~ j=1,...,J-1; \label{syst_spline2}
\end{eqnarray}
pour $j=0,J$,   
\begin{eqnarray}
\lefteqn{\Delta x_{i-1}s_{i+1,j}+2(\Delta x_{i-1}+\Delta x_i)s_{ij} 
+ \Delta x_i s_{i-1,j} = } \nonumber \\
&& 3\biggl [ \frac{\Delta x_{i-1}}{\Delta x_i}(q_{i+1,j}-q_{i,j}) 
+ \frac{\Delta x_i}{\Delta x_{i-1}}(q_{i,j}-q_{i-1,j})\biggr ],
~ i=1,...,I-1; \label{syst_spline3}
\end{eqnarray}
pour $i=0,...,I$, 
\begin{eqnarray}
\lefteqn{\Delta y_{j-1}s_{i,j+1}+2(\Delta y_{j-1}+\Delta y_j)s_{ij} 
+ \Delta y_j s_{i,j-1} = } \nonumber \\
&& 3\biggl [ \frac{\Delta y_{j-1}}{\Delta y_j}(p_{i,j+1}-p_{i,j}) 
+ \frac{\Delta y_j}{\Delta y_{j-1}}(p_{i,j}-p_{i,j-1})\biggr ],
~ j=1,...,J-1; \label{syst_spline4}
\end{eqnarray}
\end{lemme}

\subsubsection*{Algorithme d'\'evaluation de $u(x,y)$} 

\begin{description}
\item[Etape 1] Calculer les valeurs $p_{ij}$, $q_{ij}$ et $s_{ij}$ 
pour tous les points de la grille gr\^ace aux \'equations 
pr\'ec\'edentes en utilisant l'\'elimination de Gauss (les matrices 
des $2I+J-5$ syst\`emes sont tridiagonales et \`a diagonale 
strictement dominante).
\item[Etape 2] Une fois que l'on dispose des valeurs $u$, $u_x$, 
$u_y$ et $u_{xy}$ en tout point de la grille, calculer dans chaque 
rectangle $R_{ij}$ les coefficients $\gamma_{pq}^{ij}$ de la fonction 
polynomiale bicubique dans ce rectangle en utilisant (\ref{coeffs}). 
\end{description}
Evaluer la fonction d'interpolation en un point $(x,y)$ revient 
alors \`a trouver les indices $(i,j)$ tels que $(x,y) \in R_{ij}$, 
puis \`a \'evaluer la fonction polynomiale bicubique correspondante.

\subsection{Application dans l'algorithme de calibration}

Pour r\'esoudre l'EDP de Dupire, il faut conn\^{\i}tre la 
volatilit\'e $\sigma$ en tout point de la grille fine de 
diff\'erences finies de taille $N \times M$. La volatilit\'e \'etant 
d\'efinie comme une spline bicubique sur une grille grossi\`ere de 
taille $n \times m$, nous proc\'edons de fa\c con suivante pour 
obtenir la valeur de $\sigma$ en un point quelconque de la grille 
fine~: 
\begin{itemize}
\item on calcule la fonction d'interpolation par spline bicubique sur 
tout l'espace $[y_{min};y_{max}]\times [t_0;T_{max}]$ \`a partir des 
degr\'es de libert\'es sur la grille grossi\`ere $n \times m$, en 
appliquant l'algorithme d\'ecrit pr\'ec\'edemment~;
\item on \'evalue la fonction d'interpolation au point consid\'er\'e 
de la grille fine $N \times M$.
\end{itemize}
