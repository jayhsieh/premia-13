\section{Conclusion}

La m\'ethode de calibration propos\'ee se ram\`ene \`a la 
r\'esolution d'un probl\`eme inverse r\'egularis\'e. Elle se 
distingue des articles dont elle s'inspire (\cite{lag:jcf:97}, 
\cite{jac:jcf:98}, \cite{col:jcf:99}) sur plusieurs points. Le 
probl\`eme direct de pricing repose sur l'EDP de Dupire et non 
sur celle de Black-Scholes, ce qui permet d'\'evaluer la fonction 
co\^ut en une seule r\'esolution d'EDP, alors qu'il faut r\'esoudre 
autant d'EDP de Black-Scholes qu'il y a de donn\'ees pour faire 
cette m\^eme \'evaluation. Une autre particularit\'e r\'eside dans 
la repr\'esentation de $\sigma$~: nous utilisons ici des splines 
bicubiques, qui permettent de d\'efinir des fonctions $C^2$ tr\`es 
diverses. Enfin, nous avons choisi de calculer le gradient de la 
fonction co\^ut analytiquement afin d'acc\'el\'erer l'algorithme. 

L'exploitation des r\'esultats num\'eriques nous a amen\'e \`a 
faire une autre modification~: nous choisissons un coefficient 
de r\'egularisation $\lambda = 0$ et nous r\'egularisons le 
probl\`eme en utilisant une approche multi\'echelle. L'id\'ee est 
d'obtenir une premi\`ere volatilit\'e tr\`es r\'eguli\`ere sur 
une grille tr\`es grossi\`ere, puis de raffiner progressivement 
la grille de repr\'esentation de $\sigma$ pour affiner la solution. 

Les tests num\'eriques effectu\'es sur donn\'ees synth\'etiques 
ont \'egalement montr\'es qu'il \'etait important d'utiliser un 
algorithme de Quasi-Newton avec bornes pour \'eviter d'obtenir 
des valeurs n\'egatives de $\sigma$ dans certaines zones du 
domaine. Nous montrons que notre algorithme de calibration 
converge vers une solution tr\`es satisfaisante en partant 
de points initiaux (constants) tr\`es \'eloign\'es~: $0.1$, 
$0.35$, $0.5$ et $0.9$. Sur les donn\'ees r\'eelles, nous 
montrons l'int\'er\^et d'utiliser l'approche multi\'echelle 
et l'algorithme de Quasi-Newton avec bornes. La volatilit\'e 
estim\'ee est assez r\'eguli\`ere et les valeurs de $\sigma$ 
ne semblent pas aberrantes.
