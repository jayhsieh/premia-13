\documentclass[11pt]{article}
\usepackage[english]{babel}
\usepackage[latin1]{inputenc}
\usepackage{graphicx}
\usepackage{epsfig}
\usepackage[nottoc, notlof, notlot]{tocbibind}
\usepackage[T1]{fontenc}
\usepackage{amsfonts}
\newtheorem{theorem}{Theorem}[section]
\newtheorem{definition}{Definition}[section]
\newtheorem{lemma}{Lemma}[section]
\selectlanguage{english}
\title{Pricing caps on Libor rates in L�vy models}
\author{Audrey Drif\\
Inria Rocquencourt Projet MATHFI\\
email: auddrif@yahoo.fr}
\date{}
\begin{document}
\maketitle
\tableofcontents
\begin{abstract}
Here, we give an exact formula to price caps on libor rates in L�vy models. For this, we use the paper of Ernst Eberlein and Fehmi $\ddot{O}$zban\cite{Eberlein}.
\end{abstract} 
\section{Model}
Let $(\Omega,F_{T^{*}},\mathbb{P}_{T^{*}},(F_{t})_{0\leq t\leq T^{*}})$ be a complete stochastic basis, and $T_{0}<T_{1}<...<T_{n}<T_{n+1}=T^{*}$ be a discrete-tenor stucture  with $\delta =T_{i+1}-T_{i}$, i=0,...,n where $T_{i}^{*}=T^{*}-i\delta$, the libor rates is then defined by:\\
\begin{itemize}
\item For any maturity $T_{i}$ there is a bounded deterministic function $\lambda (.,T_{i})$ which represents the volatility of the forward Libor rate process $L(.,T_{i})$.
\item We assume a strictly decreasing and positive initial term structure B(0,T) (T$\in ]0,T^{*}]$). Consequentely the initial term structure L(0,T) of forward Libor rates is given for T$\in ]0,T^{*}-\delta]$ by 
$$L(0,T)=\frac{1}{\delta}(\frac{B(0,T)}{B(0,T+\delta)}-1).$$ 
\end{itemize}
Under $\mathbb{P}_{T^{*}}$, we suppose that
$$1+\delta L(t,T_{1}^{*})=(1+\delta L(0,T_{1}^{*}))exp(\int_{0}^{T_{1}^{*}} \lambda (s,T_{1}^{*})dL_{s}^{T^{*}})$$
where $$L_{t}^{T^{*}}=\int_{0}^{t} b_{s}ds + \int_{0}^{t} c_{s}^{\frac{1}{2}}dW_{s}^{T^{*}} +\int_{0}^{t} \int_{\mathbb{R}} x(\mu ^{L}-\nu^{T^{*},L})(ds,dx)$$
and $$\int_{0}^{t}\lambda(s,T_{1}^{*})b_{s}ds=-\frac{1}{2}\int_{0}^{t}c_{s}\lambda^{2}(s,T_{1}^{*})ds-\int_{0}^{t} \int_{\mathbb{R}}(e^{\lambda(s,T^{*}_{1})x}-1-\lambda(s,T^{*}_{1})x)\nu^{T^{*},L}.$$ 
$(L_{t}^{T^{*}})_{0 \leq t \leq T^{*}}$ is a non-homogeneous L�vy process, i.e. $W^{T^{*}}$ is a brownian motion, $c_{s}$ is positive such that $\int_{0}^{t^{*}} (|b_{s}|+|c_{s}|)ds < \infty$, $\mu^{L}$ is the random measure of jumps of the process, and $\nu ^{T^{*},L}(ds,dx)=F_{s}(dx)ds$ is the $\mathbb{P}_{T^{*}}$-compensator of $\mu ^{L}$. We assume that the L$\acute{e}$vy measures $F_{s}$, which are measures on $\mathbb{R}$ with $F_{s}({0})=0$ and $\int_{0}^{T^{*}} \int_{\mathbb{R}} (x^{2}\wedge 1)F_{s}(dx)ds<\infty$, satisfy the following additionnal integrability assumption 
$$\int_{0}^{T^{*}} \int_{{|x|>1}} exp(ux)F_{s}(dx)ds<\infty$$  
for $|u|\leq (1+\epsilon )M$, where M,$\epsilon$>0 are constants such that $\sum_{i=1}^{n} |\lambda(.,T_{i}^{*})|\leq M$.\\

Thus we postulate that $\forall j \in [0;n-1]$ the libor rate process can be written under $\mathbb{P}_{T_{j}^{*}}$ 
$$1+\delta L(t,T_{j+1}^{*})=(1+\delta L(0,T_{j+1}^{*}))exp(\int_{0}^{t} \lambda(s,T_{j+1}^{*})dL_{s}^{T^{*}_{j}})$$
where $$L_{t}^{T_{j}}=\int_{0}^{t} b_{s}^{T^{j}}ds + \int_{0}^{t} c_{s}^{\frac{1}{2}}dW_{s}^{T^{*}_{j}}+\int_{0}^{t}\int_{\mathbb{R}} x(\mu^{L}-\nu^{T_{j}^{*}})(ds,dx)$$
and $$\int_{0}^{t} \lambda(s,T_{j+1}^{*})b_{s}^{T_{j}^{*}}ds=-\frac{1}{2}\int_{0}^{t}c_{s}\lambda^{2}(s,T_{j+1}^{*})ds-\int_{0}^{t} \int_{\mathbb{R}}(e^{\lambda(s,T^{*}_{j+1})x}-1-\lambda(s,T^{*}_{j+1})x)\nu^{T_{j}^{*},L}.$$\\
and $$\frac{dP_{T_{j+1}^{*}}}{dP_{T_{j}^{*}}}=\varepsilon_{T^{*}_{j+1}}(\int_{0}^{t}\lambda(s,T^{*}_{j+1})c_{s}^{\frac{1}{2}}dW_{t}^{T^{*}_{j}}+\int_{0}^{t} \int_{R}(e^{\lambda(s,T_{j+1}^{*})x}-1)(\mu^{L}-\nu^{T^{*}_{j}})(ds,dx)$$\\
\\ 
and $$W_{t}^{j+1}=W_{t}^{j}-\int_{0}^{t} \lambda(s,T_{j+1}^{*})c_{s}^{\frac{1}{2}}ds$$ is a brownian motion under $\mathbb{P}_{T_{j+1}^{*}}$\\\\
and $$\nu^{T_{j+1},L}=e^{\lambda(t,T_{j+1}^{*})x}\nu^{T_{j},L}$$ is a $\mathbb{P}_{T_{j+1}^{*}}$-compensator of $\mu^{L}$\\\\
To implement in using this model, we must use the following assumption:
\begin{itemize}
\item the deterministic function $L(.,T_{i})$ is constant for $0\leq i\leq n-1$.
\end{itemize}  
\section{Pricing caps}
Now, we can get an exact formula for caps price. Indeed, let $0 \leq T_{0} \leq T_{1} \leq ... \leq T_{n}$ be a tenor structure  where $T_{i+1}-T_{i}=\delta$, the price of j-th caplet at time t is given by 
$$\mathbb{E}_{\mathbb{P}^{*}}\left[\frac{B_{t}}{B_{T_{j}}}\delta(L(T_{j-1},T_{j-1})-\mathcal{K})^{+}\Big|F_{t}\right].$$ 
Thus, the value of cap at time t$\leq T_{0}$ is given by
$$FC_{t}=\sum_{j=1}^{n}\mathbb{E}_{\mathbb{P}^{*}}\left[\frac{B_{t}}{B_{T_{j}}}\delta(L(T_{j-1},T_{j-1})-\mathcal{K})^{+}\Big|F_{t}\right].$$
For $1\leq j \leq n$, let's consider the measure $\mathbb{P}_{T_{j}}$ defined by
$$\frac{d\mathbb{P}^{*}}{d\mathbb{P}_{T_{j}}}=B_{T_{j}}B(0,T_{j}).$$
Thus, we obtain that
$$\sum_{j=1}^{n}B(t,T_{j})\mathbb{E}_{T_{j}}\left[\delta(L(T_{j-1},T_{j-1})-\mathcal{K})^{+}\Big|F_{t}\right].$$
Then we suppose that dynamics of libor rates $L(t,T_{j-1})$ under the forward measure associated with the date $T_{j-1}$ is given by

$$1+\delta L(T_{j-1},T_{j-1})=(1+\delta L(0,T_{j-1}))exp(X_{T_{j-1}})$$
where $(X_{t})$ is defined by
$$X_{t}=\int_{0}^{t}\lambda(s,T_{j-1})dL_{s}^{T_{j}}$$
and the deterministic function $L(.,T_{i})$ is constant for $0\leq i\leq n-1$.\\\\
To use the following theorem makes possible to find a formula to price caps. For this , we need the following result
 $$\mathcal{X}(u)=\mathbb{E}_{\mathbb{P}_{T{j}}}\left[e^{iuX_{T_{j-1}}}\right]=\exp\left(\int_{0}^{T{j-1}}\int_{\mathbb{R}}\left(e^{iu\lambda(s,T_{j-1})x}-iue^{\lambda(s,T_{j-1})x}-(1-iu)\right)\nu^{T_{j},L}(ds,dx)\right).$$
\begin{theorem}
Let $\zeta_{j}=-ln(1+\delta L(0,T_{j-1}))$ and $\mathcal{K}^{'}=\delta\mathcal{K}+1$. Then $L(T_{j-1},T_{j-1})=e^{-\zeta{j}+X_{T_{j-1}}}$. Assume that mgf(-R)$<\infty$ where $mgf(u)=\mathbb{E}_{T_{j-1}}\left[e^{X_{T_{j-1}}}\right]$. Let $V_{j}(\zeta_{j},\mathcal{K}^{'})$ be the time-0 price of the j-th caplet and let $\mathrm{L}[v_{\mathcal{K}^{'}}]$ be the bilateral Laplace transform of $v_{\mathcal{K}^{'}}$, i.e.\\
$$\: \mathrm{L}[v_{\mathcal{K}^{'}}]=\int_{-\infty}^{\infty} e^{-zx}v_{\mathcal{K}^{'}}(x)dx,\quad z=R+iu\in \mathbb{C}, u\in \mathbb{R}.$$ 
Then
$$V_{j}(\zeta_{j},\mathcal{K}^{'})=B(0,T_{j})\frac{e^{\zeta_{j}R}}{2\pi}\lim_{M\rightarrow\infty}\int_{M}^{M} e^{iu\zeta{_j}}\mathrm{L}[v_{\mathcal{K}^{'}}](R+iu)\mathcal{X}(iR-u)du$$
whenever the right-hand side exists. 
\end{theorem}
Then, we must give the L�vy process cases that we want to use to price caps.
\section{Program manual}
In this program, we can choose between six differents L�vy measure cases:
\begin{enumerate}
\item L�vy process where the L�vy measure is defined by 
 $$\nu=a\frac{e^{-\lambda x}}{x^{2}} \mathnormal{1}_{\{x>0\}}\quad,\lambda,a>0.$$ 
\item L�vy process where  the L�vy measure is defined by
$$\nu=a\frac{e^{-\lambda x}}{x} \mathnormal{1}_{\{x>0\}}\quad, \lambda,a>0.$$
\item L�vy process where the L�vy measure is defined by
$$\nu=a\frac{e^{-\lambda x}}{x^{\alpha +1}} \mathnormal{1}_{\{x>0\}}\quad,\:\lambda,a>0\:and\:0<\alpha<1.$$
\item L�vy process where the L�vy measure is defined by
$$\nu(x)=\frac{a_{-}}{|x|^{1+\alpha_{-}}}e^{-\lambda_{-}|x|}\mathrm{1}_{x<0}+\frac{a_{+}}{x^{1+\alpha_{+}}}e^{-\lambda_{+}x}\mathrm{1}_{x>0}$$ with $a_{+},a_{-},\lambda_{+},\lambda_{-}>0\:,\:\alpha_{-},\alpha_{+}\neq0,1\:and\:\alpha_{-},\alpha_{+}<2 $.
\item L�vy process where the L�vy measure is defined by
$$\nu(x)=\frac{a_{-}}{|x|}e^{-\lambda_{-}|x|}\mathrm{1}_{x<0}+\frac{a_{+}}{x}e^{-\lambda_{+}x}\mathrm{1}_{x>0}$$ with $a_{+},a_{-},\lambda_{+},\lambda_{-}>0$.\item L�vy process where the L�vy measure is defined by
$$\nu(x)=\frac{a_{-}}{|x|^{2}}e^{-\lambda_{-}|x|}\mathrm{1}_{x<0}+\frac{a_{+}}{x^{2}}e^{-\lambda_{+}x}\mathrm{1}_{x>0}$$ with $a_{+},a_{-},\lambda_{+},\lambda_{-}>0$. 

\end{enumerate}
When we run the program, we have to choose between this six cases and give parameters of L�vy density that we choose for this program.
\begin{itemize}
\item For the first case, we have to enter 0 and after we have to enter the parameters $a,\lambda$. 
\item For the second case, we have to enter 1 and after we have to enter the parameters $a,\lambda$.
\item For the third case, we have to enter 2 and after we have to enter the parameters $a,\lambda,\alpha$.
\item For the fourth case, we have to enter 3 and after we have to enter the parameters $a_{-},a_{+},\lambda_{-}, \lambda{+},\alpha_{-},\alpha_{+}.$
\item For the fifth case, we have to enter 4 and after we have to enter the parameters $a_{-},a_{+},\lambda_{-}, \lambda{+}.$
\item For the sixth case, we have to enter 5 and after we have to enter the parameters $a_{-},a_{+},\lambda_{-}, \lambda{+}.$\\
\end{itemize}
After to choose the L�vy density and enter this parameters, the program ask us  to enter the strike $\mathcal{K}$, the number of dates n, the date $T^{*}$, the volatility of the forward Libor rate process $L(.,T_{i})$ for $0\leq i \leq n-1$.\\\\
Now, we have to calculate bond prices. For this, we have two methods:
\begin{itemize}
\item in the first method, we use this formula $B(0,t)=\exp(-rt)$.
\item in the second method, the previous formula is wrong. Thus, we calculate bond prices by interpolation in using the values and dates of bond prices already known.\\
\end{itemize}
Thus, we have to do:
\begin{itemize}
\item if we choose the first method to calculate bond prices, the program ask us to enter the constant interest rate r.
\item if we choose the second method to calculate bond prices, the program ask us to enter the number of bond prices that we know, their dates and their values.\\  
\end{itemize}   
\newpage
\bibliographystyle{plain}\bibliography{bb}
\end{document}